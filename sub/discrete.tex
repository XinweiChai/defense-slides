\documentclass[8pt]{beamer}
\usepackage[english]{babel}
\usepackage[utf8]{inputenc}
\usepackage[T1]{fontenc}
\usepackage{lmodern}
\usetheme{Warsaw}
\useoutertheme{infolines} 
\setbeamertemplate{items}[ball]
\usepackage{algorithm}
\usepackage{fancybox}
\usepackage{hyperref}
\usepackage{tikz}
\usetikzlibrary{automata,calc,er}
\usetikzlibrary{mindmap,scopes,arrows,arrows.meta,shapes,chains,positioning,fit,backgrounds,decorations,intersections,petri,decorations.pathmorphing}
\usepackage{pgf}
\usepackage{pgfplots}
\pgfplotsset{compat=1.13}
\usetikzlibrary{pgfplots.fillbetween}
\pgfdeclarelayer{ft}
\pgfdeclarelayer{bg}
\pgfsetlayers{bg,main,ft}
\usepackage{graphics}
\usepackage{amssymb}
\usepackage{adjustbox}
\usepackage{wasysym}
\usepackage{siunitx}
\usepackage{makecell}
\usepackage{kbordermatrix}
\usepackage{mathtools}
\usepackage{calc}
\usepackage{fp}
\usepackage{docmute}
\usepackage{graphicx}
\graphicspath{{figures/}}
\input{macros}
\input{macros-ph}
\input{macros-abstr}
\input{tikzstyles2}
\usepackage{cancel}
\newcommand{\highlight}[1]{\textcolor{blue!50}{\textbf{#1}}}
\begin{document}
\begin{frame}{Discrete Modeling~\cite{bernot2009}}
\begin{columns}
\begin{column}{0.5\textwidth}
\onslide<+->{
\centering
\begin{tikzpicture}
\node (a) [draw,circle] {$a$};
\node (b) [draw,circle,right = of a] {$b$};
\draw[-{Latex[length=1.5mm]}, line width=1pt] (a) to[bend left] (b);
\path (b) edge[-|,thick,bend left=30,shorten >=1pt] (a);
\end{tikzpicture}
}

\vspace{0.5cm}
\textcolor<4->{black!20}{What are the possible values for $a$ and $b$?}
\begin{itemize}
\item<+-> \textcolor<4->{black!20}{Behaviors of elements in BRN can be approximated by sigmoid functions}
\item<+-> \textcolor<4->{black!20}{Simplified by piecewise functions with threshold.
When the concentration of $a$ is below the threshold, $a$ behaves as inhibition, otherwise activation}
\item<+-> A model containing dynamic information
\end{itemize}
\end{column}
\begin{column}{0.5\textwidth}
    \only<2-3>{\begin{tikzpicture}[scale=0.7]
\scriptsize
    \begin{axis}[samples=100,legend pos=north west,legend style={draw=none},xmin=-3,xmax=3,ymin=-0.1,ymax=1.1]
        \only<1>{\addplot[mark=none,color=red,domain= -3:3] {1/(1+exp(-2*x))};}
        \only<2>{\addplot[mark=none,color=red,domain=-3:0] {0};}
        \only<2>{\addplot[mark=none,color=red,domain=0:3] {1};}
        \only<2>{\addplot[fill=white,only marks,mark=*] coordinates{(0,0)(0,1)};}
        \only<1>{\addlegendentry{\normalsize{$f(x)=\frac{1}{1+e^{-x}}$}}}
        \only<2>{\addlegendentry{{\normalsize $f(x)$}$ =\begin{cases}0,&{\mbox{if }}x<0\\x,&{\mbox{if }}x= 1\end{cases}$}}
    \end{axis}
\end{tikzpicture}}
    \only<4->{
    
    Automata Network (AN)
    
    A modeling framework representing state transitions and using $O(\#nodes)$ memory 
    
    \vspace{0.5cm}
    \begin{tikzpicture}
\onslide<7->{
\begin{scope}[opacity=0.2]
\TSort{(1.7,0)}{b}{2}{l}
\path[local transitions]    
	(b_0) edge[bend right] node[right] {$\{a_1\}$} (b_1)
;
\end{scope}}
\TSort{(0,0)}{a}{2}{l}
\onslide<5,6>{
\TSort{(1.7,0)}{b}{2}{l}
}
\path<5->[local transitions]
	%(a_0) edge node[auto] {$\{b_1\}$} (a_1)
	(a_1) edge node[auto] {$\{b_1\}$} (a_0)
;
\path<5,6>[local transitions]
	(b_0) edge[bend right] node[right] {$\{a_1\}$} (b_1)
;
\path<7->[local transitions,opacity=0.2]
	(b_0) edge[bend right] node[right] {$\{a_1\}$} (b_1)
;
\onslide<5->{\TState{a_0,b_0}}
\onslide<7->{
\TState{a_0}
\begin{scope}[opacity=0.2]
\TState{b_0}
\end{scope}}
\end{tikzpicture}
    }
    
    \vspace{0.2cm}
    \only<5->{The global reachability is PSPACE-complete~\cite{harel2002complexity}}
    
    \only<6->{$\Longrightarrow$The complexity of reachability of $a$ can be smaller?}
    
\end{column}
\end{columns}
\end{frame}
\end{document}