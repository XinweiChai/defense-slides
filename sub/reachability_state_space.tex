\documentclass[8pt]{beamer}
\usepackage[english]{babel}
\usepackage[utf8]{inputenc}
\usepackage[T1]{fontenc}
\usepackage{lmodern}
\usetheme{Warsaw}
\useoutertheme{infolines} 
\setbeamertemplate{items}[ball]
\usepackage{algorithm}
\usepackage{fancybox}
\usepackage{hyperref}
\usepackage{tikz}
\usetikzlibrary{automata,calc,er}
\usetikzlibrary{mindmap,scopes,arrows,arrows.meta,shapes,chains,positioning,fit,backgrounds,decorations,intersections,petri,decorations.pathmorphing}
\usepackage{pgf}
\usepackage{pgfplots}
\pgfplotsset{compat=1.13}
\usetikzlibrary{pgfplots.fillbetween}
\pgfdeclarelayer{ft}
\pgfdeclarelayer{bg}
\pgfsetlayers{bg,main,ft}
\usepackage{graphics}
\usepackage{amssymb}
\usepackage{adjustbox}
\usepackage{wasysym}
\usepackage{siunitx}
\usepackage{makecell}
\usepackage{kbordermatrix}
\usepackage{mathtools}
\usepackage{calc}
\usepackage{fp}
\usepackage{docmute}
\usepackage{graphicx}
\graphicspath{{figures/}}
\input{macros}
\input{macros-ph}
\input{macros-abstr}
\input{tikzstyles2}
\usepackage{cancel}
\newcommand{\highlight}[1]{\textcolor{blue!50}{\textbf{#1}}}
\begin{document}
\begin{frame}{Reachability Problem Illustrated by Transition Graph}

\begin{columns}
\begin{column}{0.5\textwidth}
\centering
    \begin{tikzpicture}[
        > = stealth, % arrow head style
        shorten > = 1pt, % don't touch arrow head to node
        auto,
        node distance = 2cm, % distance between nodes
        semithick % line style
    ]

    \tikzstyle{every state}=[
        draw = black,
        thick,
        fill = white,
        minimum size = 4mm
    ]

    \node<1>[state,fill=blue!30] (s) {$\alpha$};
    \node<2->[state] (s) {$\alpha$};
    \node<3>[state] (v1) [above right of=s,fill=blue!30] {$v_1$};
    \node<1,2,4>[state] (v1) [above right of=s] {$v_1$};
    \node<2>[state] (v2) [right of=s,fill=blue!30] {$v_2$};
    \node<1,3->[state] (v2) [right of=s] {$v_2$};
    \node[state] (v3) [below right of=s] {$v_3$};
    \node<4>[state] (t) [right of=v2,fill=blue!30] {$\omega$};
    \node<1-3>[state] (t) [right of=v2] {$\omega$};

    \draw[->] (s) -- (v1);
    \draw[->] (s) -- (v2);
    \draw[->] (s) -- (v3);
    \draw[->] (v2) -- (v1);
    \draw[->] (v3) -- (v2);
    \draw[->,dotted] (v1) -- node [above,midway] {?} (t);
    \draw[->,dotted] (v2) -- node [above,midway] {?} (t);
    \draw[->,dotted] (v3) -- node [above,midway] {?} (t);
\end{tikzpicture}
\end{column}
\begin{column}{0.5\textwidth}
\begin{itemize}
    \item Digraph representing state space
    \begin{itemize}
        \item Nodes = system states
        \item Edges = state transitions
        \item $\alpha=$ initial state
        \item $\omega=$ desired final state
    \end{itemize}
    \item<5->Solving reachability of digraphs needs at least polynomial time and space w.r.t $\#nodes$~\cite{harel2002complexity}, but $\#nodes$ is exponential to $\#variables$

    $\Longrightarrow$ exhaustive search is not acceptable when dealing with a large model
    \item <6-> A pertinent modeling framework is necessary to describe system dynamics
\end{itemize}

\end{column}
\end{columns}
    

    
\end{frame}
\end{document}