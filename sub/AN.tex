\begin{frame}{Automata Network (AN)}
	\input{sub/figures/ANexample_onepic.tex}
	
\begin{itemize}[<+->]
	\item $\Sigma=\{a,b,c,d,e\}$ set of automata \onslide<10->{\textcolor{red}{with every automaton having a Boolean state}}
	\item States of automata: 
\begin{itemize}
    \item Local state: $a_0$: automaton $a$ is at state $0$
    \item Global state: $\langle a_0,b_0,c_0,d_0,e_0 \rangle$, the state of the whole system
    \item Joint state: $\{a_0,b_0\}$ a part of the system state 
\end{itemize}
 
\item Transition:
\begin{itemize}
    \item $\{b_1,c_1\}\to a_0\Rsh a_1$: $a$ can transit from state $0$ to state $1$ when joint state $\{b_1,c_1\}$ is present
\end{itemize}
\item Update scheme
\begin{itemize}
    \item Asynchronous Automata Network (AAN)~\cite{folschette2015}: at most one transition can be fired at each time point which conform to biological non-deterministic dynamics
\end{itemize}
%For transition $tr=A\to b_i$, $A$ (called head, noted $head(tr)$) is the set of required state(s), which allows to flip $b_{1-i}$ to $b_i$ (called body, noted $body(tr)$). In other words, transition $tr$ is said fireable iff $A\subseteq s$, where $s$ is the current global state. 
\item To simplify further reachability analysis, we limit the AAN to Boolean values
	
	$\to$ Asynchronous Binary Automata Network (ABAN)
    
    The notation of transitions is simplified to $\{b_1,c_1\}\to a_1$
\item To study reachability problem $\to$ Simplified Local Causality Graph (SLCG) based on Local Causality Graph (LCG)~\cite{pauleve2012}
\end{itemize}
	
\end{frame}