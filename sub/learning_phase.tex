\documentclass[8pt]{beamer}
\usepackage[english]{babel}
\usepackage[utf8]{inputenc}
\usepackage[T1]{fontenc}
\usepackage{lmodern}
\usetheme{Warsaw}
\useoutertheme{infolines} 
\setbeamertemplate{items}[ball]
\usepackage{algorithm}
\usepackage{fancybox}
\usepackage{hyperref}
\usepackage{tikz}
\usetikzlibrary{automata,calc,er}
\usetikzlibrary{mindmap,scopes,arrows,arrows.meta,shapes,chains,positioning,fit,backgrounds,decorations,intersections,petri,decorations.pathmorphing}
\usepackage{pgf}
\usepackage{pgfplots}
\pgfplotsset{compat=1.13}
\usetikzlibrary{pgfplots.fillbetween}
\pgfdeclarelayer{ft}
\pgfdeclarelayer{bg}
\pgfsetlayers{bg,main,ft}
\usepackage{graphics}
\usepackage{amssymb}
\usepackage{adjustbox}
\usepackage{wasysym}
\usepackage{siunitx}
\usepackage{makecell}
\usepackage{kbordermatrix}
\usepackage{mathtools}
\usepackage{calc}
\usepackage{fp}
\usepackage{docmute}
\usepackage{graphicx}
\graphicspath{{figures/}}
\input{macros}
\input{macros-ph}
\input{macros-abstr}
\input{tikzstyles2}
\usepackage{cancel}
\newcommand{\highlight}[1]{\textcolor{blue!50}{\textbf{#1}}}

\usepackage{array}
\usepackage{booktabs}
\begin{document}
\begin{frame}{Learning Phase}
CRAC: using correlation coefficient $r_{x,y}={\frac {\operatorname {cov} (x,y)}{\sigma _{x}\sigma _{y}}}$\pause$={\frac {\sum _{i=1}^{N}(x_{i}-{\bar {x}})(y_{i}-{\bar {y}})}{{\sqrt {\sum _{i=1}^{N}(x_{i}-{\bar {x}})^{2}}}{\sqrt {\sum _{i=1}^{N}(y_{i}-{\bar {y}})^{2}}}}}$

\pause
\vspace{0.2cm}
\begin{tabular}{ll}
1.\begin{tabular}{l|*{4}{l}}
$t$&0&1&2&3\\
\hline
$a$&2.01&2.51&1.97&1.17\\
$b$&0.74&0.87&0.78&0.33\\
\end{tabular}   
\pause
& 
2.
\begin{tabular}{c|c*{3}{S}}
$t$&1&2&3\\
\hline
$\Delta a$&0.5&-0.54&-0.8\\
$\Delta b$&0.13&-0.09&-0.45\\
\end{tabular}
\pause
\\
3.
$r'=\kbordermatrix{\mbox{}&\Delta a&\Delta b\\
a&\text{N/A}&0.09\\
b&0.65&\text{N/A}\\
}$ 
\pause
&
4.
\begin{tikzpicture}[grn]
    \path[use as bounding box] (-0.3,1) rectangle (4,0);
      \node[inner sep=0] (b) at (2,0) {b};
      \node[inner sep=0] (a) at (0,0) {a};
      \path[->]
       (b) edge[bend right] (a);
\end{tikzpicture}
\end{tabular}

\vspace{0.5cm}
\pause

M2RIT: using LFIT algorithm

\vspace{0.2cm}
\begin{tabular}{ccc}
    Full transitions obtained from time-series data & classed into & partial transitions\\
    $\langle a_1,b_0,c_1\rangle(t=T)\to \langle a_1,b_1,c_1\rangle(t=T+1)$&&\\
    $\langle a_1,b_0,c_0\rangle(t=T)\to \langle a_1,b_1,c_0\rangle(t=T+1)$&$\Longrightarrow$&$\{a_1\}\to b_1$\\
    $\langle a_0,b_0,c_0\rangle(t=T)\to \langle a_0,b_0,c_1\rangle(t=T+1)$&&
\end{tabular}

\vspace{0.2cm}
\pause
The classification is sensitive to input $\to$ revise the result
\end{frame}
\end{document}