\documentclass[8pt]{beamer}
\usepackage[english]{babel}
\usepackage[utf8]{inputenc}
\usepackage[T1]{fontenc}
\usepackage{lmodern}
\usetheme{Warsaw}
\useoutertheme{infolines} 
\setbeamertemplate{items}[ball]
\usepackage{algorithm}
\usepackage{fancybox}
\usepackage{hyperref}
\usepackage{tikz}
\usetikzlibrary{automata,calc,er}
\usetikzlibrary{mindmap,scopes,arrows,arrows.meta,shapes,chains,positioning,fit,backgrounds,decorations,intersections,petri,decorations.pathmorphing}
\usepackage{pgf}
\usepackage{pgfplots}
\pgfplotsset{compat=1.13}
\usetikzlibrary{pgfplots.fillbetween}
\pgfdeclarelayer{ft}
\pgfdeclarelayer{bg}
\pgfsetlayers{bg,main,ft}
\usepackage{graphics}
\usepackage{amssymb}
\usepackage{adjustbox}
\usepackage{wasysym}
\usepackage{siunitx}
\usepackage{makecell}
\usepackage{kbordermatrix}
\usepackage{mathtools}

\usepackage{calc}
\usepackage{fp}
\input{macros}
\input{macros-ph}
\input{macros-abstr}
\input{tikzstyles2}
\usepackage{cancel}
\newcommand{\highlight}[1]{\textcolor{blue!50}{\textbf{#1}}}
\title[Reachability Analysis and Revision of Dynamics]{Reachability Analysis and Revision of Dynamics of\\ Biological Regulatory Networks}
\author[X.Chai]{Xinwei Chai}
\institute[LS2N]{
Le Laboratoire des Sciences du Numérique de Nantes\\
École Centrale de Nantes\\
\texttt{xinwei.chai@ls2n.fr}

\vspace{1cm}
\begin{tabular}{r@{\ \ }l}
\textbf{Rapporteurs :}
& Gilles BERNOT, Professeur des universités,
    Université Côte d'Azur \\
& Pascale LE GALL, Professeur des universités,
    Centrale Supélec \vspace*{1em} \\
\textbf{Examinateurs :}
& Béatrice DUVAL, Professeur des universités, Université d'Angers  \\
& Loïc PAULEVÉ, Chargé de recherche,
    LaBRI, UMR CNRS \vspace*{1em} \\
\textbf{Directeur de thèse :}
& Olivier ROUX, Professeur des universités,
    École Centrale de Nantes \\
\textbf{Co-encadrant de thèse :}
& Morgan MAGNIN, Professeur des universités,
    École Centrale de Nantes
\end{tabular}
}
\date[May 24, 2019]{May 24, 2019}
\begin{document}

\begin{frame}[plain]
  \titlepage
\end{frame}

\begin{frame}{Positioning of Our Work}
\begin{adjustbox}{max totalsize={.9\textwidth}{.9\textheight},center}
    \centering
    \begin{tikzpicture}[mindmap, scale=1.13,
    level 1 concept/.append style={level distance=60,sibling angle=30,font=\large},
    extra concept/.append style={color=blue!50,text=black}, every node/.style={scale=0.5}]
    \begin{scope}[mindmap, concept color=blue,text=white]
            \onslide<1->{\node [concept] (bioapp) at (2,3.4) {Biological Applications} [counterclockwise from=30] 
            child{node [concept] (bioen) {Bio-engineered livers}}
            child{node [concept] (int) {Inter-cellular signaling}}
            child{node [concept] (env) {Environment toxicology}}
            child{node [concept] (inf) {Infectious diseases}}
            child{node [concept] (cell) {Cell cycle}};}
    \end{scope}

    \begin{scope}[mindmap, concept color=orange, text=white,font=\large]
        \onslide<2->{\node [concept] (exp) {Experimental Approaches}[counterclockwise from=90] 
            child{node [concept] (dna) {DNA microarrays}}
            child{node [concept] (prot) {Proteomics}}
            child{node [concept] (rtms) {Real-time mass spectroscopy}}
            child{node [concept] (mic) {Microfluidics}};}
    \end{scope}

    \begin{scope}[mindmap, concept color=red,text=white,font=\large]
        \onslide<4->{\node [concept] (mod) at (4,0) {Modeling and Computation}[counterclockwise from=7.5] 
            child{node [concept] (net) {Network biology}}
            child{node [concept] (csm) {Predictive models}}
            child{node [concept] (gra) {Graph theory}}
            child{node [concept] (sim) {Simulation}};}
    \end{scope}

    % Connections of researchers to applied subfields

    \begin{pgfonlayer}{bg}
        \draw<3-> [circle connection bar]
            (exp) edge (bioapp);
        \draw<5-> [circle connection bar]
            (bioapp) edge (mod);
        \draw<6-> [circle connection bar]
            (mod) edge (exp);
    \end{pgfonlayer}
        
    \onslide<7>{\node (mark) at (5.5,4.5) [text width=4cm, inner sep=5pt,minimum size=5pt]  {\Huge{Informatic part}};}
    \only<8>{\node (mark) at (5.5,4.5) [text width=4cm, inner sep=5pt,minimum size=5pt]  {\Huge{Our work is here}};}
    \onslide<7->{%\node (ellip) at (4.5,0.8) {};
        \coordinate (ellip) at (4.4,0.8);
        \draw[thick, draw=black] (ellip) ellipse (2.4 and 2);
      \draw[->,very thick, bend left=15] (mark) edge[->] (5.3,3);}

\end{tikzpicture}
\end{adjustbox}
\end{frame}

\begin{frame}{Biological Regulatory Networks}
    \begin{tikzpicture}[-{Latex[length=1.5mm]}]
  \onslide<1->{\node[inner sep=0pt] (cell) {\includegraphics[width=0.45\textwidth]{cell.png}};}
  \onslide<3->{\node[right = of cell] (rnaA)  {\scriptsize RNA of $a$};}
  \onslide<2->{\node[above = of rnaA] (dnaA) {\scriptsize DNA of gene $a$};}
  \onslide<1->{\draw[-latex,thick] (cell) -- (rnaA);}
  \onslide<4->{\node[below = of rnaA] (prA) {\scriptsize Protein of $a$};}
  \onslide<2->{\node[right = of dnaA] (dnaB) {\scriptsize DNA of gene $b$};}
  \onslide<5->{\node[below = of dnaB] (rnaB) {\scriptsize RNA of $b$};}
  \onslide<6->{\node[below = of rnaB] (prB) {\scriptsize Protein of $b$};}
  \onslide<3->{\draw (dnaA) to node[midway,label=left:{\scriptsize Transcription}]{} (rnaA);}
  \onslide<4->{\draw (rnaA) to node[midway,label=left:{\scriptsize Translation}]{} (prA);}
  \onslide<5->{\draw (dnaB) to node[midway,label=right:{\scriptsize Transcription}]{} (rnaB);}
  \onslide<6->{\draw (rnaB) to node[midway,label=right:{\scriptsize Translation}]{} (prB);}
  \onslide<9->{\node[draw,dotted,fit=(dnaA) (rnaA) (prA),label=above:$a$] (entityA){};}
  \onslide<10->{\node[draw,dotted,fit=(dnaB) (rnaB) (prB),label=above:$b$] (entityB){};}
  \onslide<7->{\draw[bend left=15,color=red] (prA) to (dnaB);}
  \onslide<8->{\draw[-|,bend right=15, color=blue!50] (prB) to (dnaA);}
  %\onslide<9->{\draw[bend right=45, color=blue!50,dashed] (prB) to (dnaB);}
\onslide<12->{
\node (a) [draw,circle,below left = 2cm and 1cm of entityA] {$a$};
\node (b) [draw,circle,right = of a] {$b$};
\draw[-{Latex[length=1.5mm]}, line width=1pt] (a) to[bend left] (b);
\path (b) edge[-|,thick,bend left=30,shorten >=1pt] (a);
%\draw[-{Latex[length=1.5mm]}, thick, loop above,dashed] (b) to (b);
}
\node[fit=(a)(b)](mid){};
\onslide<11->{\draw[-latex,thick] (prA) -- (mid);}
\onslide<13->{\node [draw=none, right = of b,text width=3cm]{How to analyze system dynamics using BRN?};}
\end{tikzpicture}
\end{frame}

\begin{frame}{Discrete Modeling}
\begin{columns}
\begin{column}{0.5\textwidth}
   \begin{tikzpicture}[scale=0.7]
\scriptsize
    \begin{axis}[samples=100,legend pos=north west,legend style={draw=none},xmin=-3,xmax=3,ymin=-0.1,ymax=1.1]
        \only<1>{\addplot[mark=none,color=red,domain= -3:3] {1/(1+exp(-2*x))};}
        \only<2>{\addplot[mark=none,color=red,domain=-3:0] {0};}
        \only<2>{\addplot[mark=none,color=red,domain=0:3] {1};}
        \only<2>{\addplot[fill=white,only marks,mark=*] coordinates{(0,0)(0,1)};}
        \only<1>{\addlegendentry{\normalsize{$f(x)=\frac{1}{1+e^{-x}}$}}}
        \only<2>{\addlegendentry{{\normalsize $f(x)$}$ =\begin{cases}0,&{\mbox{if }}x<0\\x,&{\mbox{if }}x= 1\end{cases}$}}
    \end{axis}
\end{tikzpicture}
\end{column}
\begin{column}{0.5\textwidth}
\begin{itemize}
\item<+-> Behaviors of elements in BRN can be approximated by sigmoid functions
\item<+-> Simplified by piecewise functions with threshold.
When the concentration of $x$ is below the threshold, $x$ behaves as inhibition, otherwise promotion
\end{itemize}
\end{column}
\end{columns}
\end{frame}

\section{Motivation}
\begin{frame}{Problematic of Reachability Problem}
    \centering
    \begin{tikzpicture}
    \onslide<1->{\node[ellipse, fill=blue!20] (dynamics) at (0, 0) {System Dynamics};
\node (a) [draw,circle,left = 3cm of dynamics] {$a$};
\node (b) [draw,circle,right = of a] {$b$};
\draw[-{Latex[length=1.5mm]}, line width=1pt] (a) to[bend left] (b);
\path (b) edge[-|,thick,bend left=30,shorten >=1pt] (a);
}
\footnotesize
    \onslide<2->{\node[ellipse, fill=blue!20, below = of dynamics] (reach) {Reachability problem};
    \draw[->,thick,dashed] (dynamics) -- (reach);}
    \onslide<3->{\node[ellipse, fill=purple!20, below left = 0.6cm of reach] (full) {Exhaustive analysis};
    \draw[->,thick] (reach) -- (full);}
    \onslide<5->{\node[ellipse, fill=purple!20, below right = 0.6cm of reach] (part) {Partial analysis};
    \draw[->,thick] (reach) -- (part);}
    \onslide<4->{\node[ellipse, fill=purple!20, below = 0.6cm of full, text width = 3cm] (propfull) {Explore the whole state space $\to$ exponential \frownie{}};
    \draw[->,thick] (full) -- (propfull);}
    \onslide<6->{\node[ellipse, fill=purple!20, below = 0.6cm of part, text width = 3cm] (proppart) {Explore a part of the state space $\to$ computable \smiley{}};
    \draw[->,thick] (part) -- (proppart);}
    \onslide<7->{\node[ellipse, fill=purple!20, below = 0.6cm of proppart, text width = 3cm] (incomp) {Incomplete search \frownie{}};
    \draw[->,thick] (proppart) -- (incomp);}
    \onslide<8->{\node[draw, dashed, color = blue!50, below = 0.6cm of propfull, text width = 3cm] (heu) {Maybe some heuristics in non searched area?};
    \draw[->,dashed, thick, color=purple!30] (incomp) -- (heu);
    \draw[->,dashed, thick, color=purple!30] (propfull) -- (heu);}
    %\draw[thick,blue,rounded corners=1mm,dashed,fill=gray!40,opacity=0.2] (heu) \irregularcircle{3cm}{3mm};
\end{tikzpicture}
\end{frame}

\begin{frame}{Challenge and Solution}
    \centering
    \newcommand\irregularcircle[2]{% radius, irregularity
  \pgfextra {\pgfmathsetmacro\len{(#1) - 0.6431795905004605*(#2)}}
  +(0:\len pt)
  \foreach \a / \b in {10 / -0.05472288618682586,
20 / 0.8450075432043147,
30 / -0.27472087164920134,
40 / 0.46170068673919884,
50 / -0.2723285608995809,
60 / -0.4556287892417712,
70 / 0.29673989520725863,
80 / -0.37119286455820033,
90 / -0.15192938503079412,
100 / 0.49896661010967835,
110 / -0.3398657017968927,
120 / 0.5966632011999364,
130 / -0.36408890412829553,
140 / 0.07152555046450693,
150 / 0.18360785381458844,
160 / 0.6988961369997699,
170 / 0.07277114122898687,
180 / -0.2946202695,
190 / 0.12114187507333307,
200 / 0.7101010327107955,
210 / -0.6195271671767566,
220 / 0.9892295209020814,
230 / -0.47131882061101593,
240 / -0.5056522718446919,
250 / -0.3851643364766226,
260 / 0.9105042125287848,
270 / 0.8666611821684194,
280 / 0.08511851734790343,
290 / 0.187469288526535,
300 / 0.09433763143730722,
310 / -0.8498584485355265,
320 / -0.9989976841965511,
330 / -0.9708632213653312,
340 / 0.6032242990652343,
350 / 0.629277315289333
} {
    \pgfextra {\pgfmathsetmacro\len{(#1)+\b*(#2)}}
    -- +(\a:\len pt)
  } -- cycle
}
\begin{tikzpicture}[scale=0.8]
  \coordinate (c) at (-0.2,0);
  \uncover<1->{\draw[name path = A,thick,purple,rounded corners=1mm,dashed,fill=blue!10] (c) \irregularcircle{3cm}{3mm};}
  %\draw<1->[thick,blue,rounded corners=1mm,dashed,fill=gray!40,opacity=0.2] (c) ellipse (3.5 and 2.5);
  %\draw[blue] (-1.2,-1.2) -- (1.2,-1.2) -- (1.2,1.2) -- (-1.2,1.2) -- (-1.2,-1.2);
  \onslide<3->{\draw[name path = C,thick,purple,fill=yellow!10] (-2.5,-1.2) rectangle (2.3,1.2);}
  \onslide<2->{\draw[name path = B,thick,purple] (-7,-3.5) rectangle (3.5,3.5);}
  \only<2->{\tikzfillbetween[of=A and B, on layer=bg]{yellow!10};}
  %\onslide<3->{\tikzfillbetween[of=A and C]{yellow!40,opacity=0.2};}
  \onslide<3->{\node (under) at (0,0) {Under-approximation};}
  \onslide<2->{\node [text width=3cm] (over) at (-4.8,0) {Over-\\approximation~\cite{pauleve2012}};}
  \onslide<1>{\node (real) at (0,2) {Exact solution};}
  \onslide<4->{\node (real) at (0,2) {Inconclusive area};}
\end{tikzpicture}
    
    \vspace{0.5cm}
    \onslide<5>{$\to$ Apply heuristics on inconclusive area}
\end{frame}

\begin{frame}{Problematic of Model Inference}
    \centering
        \begin{tikzpicture}[line,>=stealth]
    \small
        \onslide<1->{\node [color=gray] (1) {Real world};}
        \onslide<2->{\node [below = 4cm of 1] (3) {Partial observation};}
        \onslide<2->{\draw [dashed,->] (1) -- (3);}
        \onslide<3->{\node [right = of 3] (4) {\textit{Learning}};}
        \onslide<3->{\draw [->] (3) -- (4);}
        \onslide<7->{\node [color=blue,right = of 1] (2) {Temporal properties};}
        \onslide<8->{\node [color=blue,right = of 2] (5) {Reachability};}
        \onslide<4->{\node [below = 3.95cm of 5] (8) {Model};}
        \onslide<4->{\draw [->] (4) -- (8);}
        \draw<5>[->,thick,dashed,color=red] (8) to node[color=black] {\LARGE X} (1);
        \node<5>[above left = 2cm and -1cm of 8, text width = 3cm, color=blue]{Inconsistent with real world behaviors};
        \onslide<6->{\node [color=blue,above = of 2] (6) {Biological \textit{a priori} knowledge};}
        \onslide<6->{\draw [dashed,->] (1) -- (6);}
        \onslide<8->{\draw [color=blue,->] (2) -- (5);}
        \onslide<7->{\draw [color=blue, ->] (6) -- (2);}
        \onslide<9->{\node [color=red, above = 2cm of 4] (7) {\textit{Model Checking}};}
        \onslide<10->{\node [draw, circle, color = red, above = 2cm of 8] (9) {$+$};}
        \onslide<10->{\draw [thick, color=red] (4) --(9);}
        \onslide<9->{\draw [->] (5) --(7);}
        \onslide<10->{\draw [thick, color=red] (7) --(9);}
        \onslide<11->{\draw [thick, color=red] (9)--(8);}
    \end{tikzpicture}
\end{frame}

\begin{frame}{Reachability Problem Illustrated by Petri Nets}
    \centering
    \begin{tikzpicture}[node distance=1.3cm,>=stealth',bend angle=45,auto]

  \tikzstyle{place}=[circle,thick,draw=blue!75,fill=blue!20,minimum size=6mm]
  \tikzstyle{red place}=[place,draw=red!75,fill=red!20]
  \tikzstyle{transition}=[rectangle,thick,draw=black!75,
  			  fill=black!20,minimum size=4mm]

  \tikzstyle{every label}=[red]

  \begin{scope}
    % First net
    \onslide<1,4>{\node [place,tokens=1] (w1)                                    {};
    \node [place] (c1) [below of=w1]                      {};
    \node [place] (s)  [below of=c1] {};
    \node [place] (c2) [below of=s]                       {};
    \node [place,tokens=1] (w2) [below of=c2]                      {};}
    \onslide<2>{\node [place] (w1)                                    {};
    \node [place] (c1) [below of=w1,tokens=1]                      {};
    \node [place] (s)  [below of=c1,tokens=1] {};
    \node [place] (c2) [below of=s]                       {};
    \node [place,tokens=1] (w2) [below of=c2]                      {};}
    \onslide<3>{\node [place] (w1)                                    {};
    \node [place,tokens=1] (c1) [below of=w1]                      {};
    \node [place,tokens=2] (s)  [below of=c1] {};
    \node [place,tokens=1] (c2) [below of=s]                       {};
    \node [place] (w2) [below of=c2]                      {};}

    \node<2> [transition] (e1) [left of=c1] {}
      edge [pre,bend left,color=red]                  (w1)
      edge [post,bend right,color=red]                (s)
      edge [post,color=red]                           (c1);
    \node<1,3,4> [transition] (e1) [left of=c1] {}
      edge [pre,bend left]                  (w1)
      edge [post,bend right]                (s)
      edge [post]                           (c1);

    \node<1,2,4> [transition] (e2) [left of=c2] {}
      edge [pre,bend right]                 (w2)
      edge [post,bend left]                 (s)
      edge [post]                           (c2);
      
    \node<3> [transition] (e2) [left of=c2] {}
      edge [pre,bend right,color=red]                 (w2)
      edge [post,bend left,color=red]                 (s)
      edge [post,color=red]                           (c2);

    \node [transition] (l1) [right of=c1] {}
      edge [pre]                            (c1)
      edge [pre,bend left]                  (s)
      edge [post,bend right] node[swap] {2} (w1);

    \node [transition] (l2) [right of=c2] {}
      edge [pre]                            (c2)
      edge [pre,bend right]                 (s)
      edge [post,bend left]  node {2}       (w2);
  \end{scope}
  
  \begin{scope}[xshift=6cm]
    % First net
    \node [place] (w1')                                    {};
    \node [place,tokens=1] (c1') [below of=w1']                      {};
    \node [place,tokens=2] (s')  [below of=c1'] {};
    \node [place,tokens=1] (c2') [below of=s']                       {};
    \node [place] (w2') [below of=c2']                      {};

    \node [transition] (e1') [left of=c1'] {}
      edge [pre,bend left]                  (w1')
      edge [post,bend right]                (s')
      edge [post]                           (c1');

    \node [transition] (e2') [left of=c2'] {}
      edge [pre,bend right]                 (w2')
      edge [post,bend left]                 (s')
      edge [post]                           (c2');

    \node [transition] (l1') [right of=c1'] {}
      edge [pre]                            (c1')
      edge [pre,bend left]                  (s')
      edge [post,bend right] node[swap] {2} (w1');

    \node [transition] (l2') [right of=c2'] {}
      edge [pre]                            (c2')
      edge [pre,bend right]                 (s')
      edge [post,bend left]  node {2}       (w2');
  \end{scope}

  \onslide<1,2>{\draw [-to,thick,decorate, decoration=snake, segment length=3mm]
    ([xshift=5mm]s -| l1) -- ([xshift=-5mm]s' -| e1')
    node [above=1mm,midway,text width=3cm,text centered]
      {Reachable?};}
  \onslide<3>{\draw [-to,thick,decorate, decoration=snake, segment length=3mm]
    ([xshift=5mm]s -| l1) -- ([xshift=-5mm]s' -| e1')
    node [above=1mm,midway,text width=3cm,text centered]
      {Reachable.};}
  \onslide<4>{\node (alpha) at ([xshift=6mm]s -| l1) {\huge$\alpha$};
  \node (alpha) at ([xshift=-6mm]s' -| e1') {\huge$\omega$};}
  \begin{pgfonlayer}{bg}
    \filldraw [line width=4mm,rounded corners,black!10]
      (w1.north  -| l1.east)  rectangle (w2.south  -| e1.west)
      (w1'.north -| l1'.east) rectangle (w2'.south -| e1'.west);
      
  \end{pgfonlayer}
\end{tikzpicture}    
\end{frame}

\begin{frame}{Reachability Problem Illustrated by State Space}
Every system state is represented by a node in the digraph

    \vspace{0.2cm}
    {\centering
    \begin{tikzpicture}[
        > = stealth, % arrow head style
        shorten > = 1pt, % don't touch arrow head to node
        auto,
        node distance = 2cm, % distance between nodes
        semithick % line style
    ]

    \tikzstyle{every state}=[
        draw = black,
        thick,
        fill = white,
        minimum size = 4mm
    ]

    \node<1>[state,fill=blue!30] (s) {$\alpha$};
    \node<2->[state] (s) {$\alpha$};
    \node<3>[state] (v1) [above right of=s,fill=blue!30] {$v_1$};
    \node<1,2,4>[state] (v1) [above right of=s] {$v_1$};
    \node<2>[state] (v2) [right of=s,fill=blue!30] {$v_2$};
    \node<1,3->[state] (v2) [right of=s] {$v_2$};
    \node[state] (v3) [below right of=s] {$v_3$};
    \node<4>[state] (t) [right of=v2,fill=blue!30] {$\omega$};
    \node<1-3>[state] (t) [right of=v2] {$\omega$};

    \draw[->] (s) -- (v1);
    \draw[->] (s) -- (v2);
    \draw[->] (s) -- (v3);
    \draw[->] (v2) -- (v1);
    \draw[->] (v3) -- (v2);
    \draw[->,dotted] (v1) -- node [above,midway] {?} (t);
    \draw[->,dotted] (v2) -- node [above,midway] {?} (t);
    \draw[->,dotted] (v3) -- node [above,midway] {?} (t);
\end{tikzpicture}
    
    \vspace{0.2cm}
    $\alpha=$ initial state, $\omega=$ desired final state}
    
\end{frame}

\begin{frame}{Modeling System Dynamics}
    \begin{columns}
    \begin{column}{0.45\textwidth}
    Boolean Network (BN)
    
    \vspace{0.5cm}
    \includegraphics[width=\textwidth]{figures/BooleanNetwork.png}
    \end{column}
    \pause
    \begin{column}{0.45\textwidth}
    A modeling framework representing state transitions and using $O(n)$ memory $\to$ Automata Network (AN)
    
    \pause
    \vspace{0.5cm}
    \begin{tikzpicture}
\onslide<7->{
\begin{scope}[opacity=0.2]
\TSort{(1.7,0)}{b}{2}{l}
\path[local transitions]    
	(b_0) edge[bend right] node[right] {$\{a_1\}$} (b_1)
;
\end{scope}}
\TSort{(0,0)}{a}{2}{l}
\onslide<5,6>{
\TSort{(1.7,0)}{b}{2}{l}
}
\path<5->[local transitions]
	%(a_0) edge node[auto] {$\{b_1\}$} (a_1)
	(a_1) edge node[auto] {$\{b_1\}$} (a_0)
;
\path<5,6>[local transitions]
	(b_0) edge[bend right] node[right] {$\{a_1\}$} (b_1)
;
\path<7->[local transitions,opacity=0.2]
	(b_0) edge[bend right] node[right] {$\{a_1\}$} (b_1)
;
\onslide<5->{\TState{a_0,b_0}}
\onslide<7->{
\TState{a_0}
\begin{scope}[opacity=0.2]
\TState{b_0}
\end{scope}}
\end{tikzpicture}
    
    \onslide<3->{The reachability of $(x,y,z)$ is of $O(2^n)$ $\Longrightarrow$}
    
    \onslide<4->{The reachability of $x$ can be of $O(n)$}
    
    \onslide<5->{$\Longrightarrow$ Local Causality Graph}
    \end{column}
    \end{columns}
\end{frame}

\begin{frame}{Local Causality Graph}
\begin{tikzpicture}[scale=0.95]
\TSort{(0,0)}{a}{2}{l}
\TSort{(2,0)}{b}{2}{l}
\TSort{(4,0)}{c}{2}{l}
\TSort{(6,0)}{d}{2}{l}
\TSort{(8,0)}{e}{2}{l}

% with delays
\path[local transitions]

	(d_0) edge node[auto] {\textcolor<10>{red}{$\{b_1$\}}} (d_1)
    (a_0) edge[bend left] node[auto] {\textcolor<3,5,6>{red}{$\{b_1,c_1\}$}} (a_1)
    (a_0) edge[bend right] node[right] {\textcolor<2,4>{red}{$\{e_1$\}}} (a_1)
    (b_0) edge[bend left] node[auto] {\textcolor<7>{red}{$\{d_0\}$}} (b_1)
	(c_0) edge node[auto] {\textcolor<8>{red}{$\{d_1\}$}} (c_1)
%	(d_0) edge node[auto] {$b_1$} (d_1)
	%(c_1) edge node[auto] {$\{b_0\}$, $2$} (c_0)
	%(c_0) edge node[auto] {$d_1$} (c_1)
;

\onslide<1-10>{\TState{a_0, b_0, c_0, d_0,e_0}}
\onslide<11>{\TState{a_0, b_1, c_0, d_0,e_0}}
\onslide<12>{\TState{a_0, b_1, c_0, d_1,e_0}}
\onslide<13>{\TState{a_0, b_1, c_1, d_1,e_0}}
\onslide<14->{\TState{a_1, b_1, c_1, d_1,e_0}}
\end{tikzpicture}

\begin{tikzpicture}[aS,scale=0.9, every node/.style={scale=0.9}]  
  	\onslide<1->{\node[color=red,Aproc] (a_1) {$a_1$};}
  	\onslide<3->{\node[Asol,right of=a_1] (a_1s) {};}
  	\onslide<3->{\path (a_1) edge (a_1s);}
  	%\onslide<10->{\node[right = 0.1cm of a_1s] {\textcolor<10>{red}{AND}};}
  	%\onslide<11->{\node[above = 0.1cm of a_1] {\textcolor<11>{red}{OR}};}
    \onslide<2->{\node[Asol,left of=a_1] (a_1s1){};}
    \onslide<4->{\node[Aproc,left of=a_1s1] (e_1){$e_1$};}
    \onslide<4->{\path (a_1s1) edge (e_1);}
    \onslide<2->{\path(a_1) edge (a_1s1);}
  	\onslide<5->{\specl{above}{a_1}{b_1};}
  	\onslide<7->{\link{b_1}{d_0};}
  	\onslide<9->{\edl{d_0};}
  	\onslide<6->{\specl{below}{a_1}{c_1};}
	\onslide<8->{\link{c_1}{d_1};}
    \onslide<10->{\path (d_1s) edge (b_1);}
    \onslide<11>{\node[draw,thick,dotted,fit=(b_1)(d_0),color=red!70,ellipse]{};}
    \onslide<12>{\draw[rotate=140,thick,dotted,color=red!70]($(b_1)!1/2!(d_1)$) ellipse (2cm and 0.45cm);}
    \onslide<13>{\node[draw,thick,dotted,fit=(c_1)(d_1),color=red!70,ellipse]{};}
    \onslide<14>{\node[draw,thick,dotted,fit=(b_1)(c_1)(a_1),color=red!70,ellipse]{};}
\end{tikzpicture}


\only<1-5>{Small circles stand for transition nodes, squares for state nodes}
\begin{columns}
\begin{column}{0.5\textwidth}
\begin{align*}
    \only<6->{r'(a_1)&=r'(e_1)\lor (r'(b_1)\land r'(c_1)\\}
    \only<7->{&=r'(d_0)\land r'(c_1)\\}
    \only<8->{&=r'(d_0)\land r'(d_1)\\}
    \only<9->{&=r'(d_1)\\}
    \only<10->{&=r'(b_1)=r'(d_0)=1}
\end{align*}
\end{column}
\begin{column}{0.5\textwidth}
    \only<15>{However, $r'(a_i)$ is \highlight{not equivalent} to the reachability of $a_1$}
\end{column}
\end{columns}
\end{frame}

\begin{frame}{Counterexample of LCG}
LCG is \highlight{exact} for unreachability 

LCG is \highlight{exact} when it does not contain self-dependent structure:
\begin{itemize}
\item different state nodes of the same automaton in different branches:
\end{itemize}
Given initial state: $a_0,b_0,c_0$, consider the reachability of $c_1$:

\begin{columns}
\begin{column}{0.45\textwidth}
    \begin{tikzpicture}[every node/.style={scale=0.9}]
\TSort{(0,0)}{a}{2}{l}
\TSort{(1.7,0)}{b}{2}{l}
\TSort{(3.7,0)}{c}{2}{l}
\path[local transitions]

	(c_0) edge node[auto] {\{$a_1, b_1$\}} (c_1)
    (a_0) edge[bend left] node[auto] {$\{b_0\}$} (a_1)
	(b_0) edge node[auto] {$\{\mathbf{c_0}\}$} (b_1)
;

\TState{a_0, b_0, c_0}


\end{tikzpicture}

\end{column}
\begin{column}{0.55\textwidth}
    \begin{tikzpicture}[aS]  
  	
  	  	\node[color=red,Aproc] (c_1) {$c_1$};\node[Asol,right of=c_1] (c_1s) {};\path (c_1) edge (c_1s);
  	\specl{above}{c_1}{a_1};
  	\link{a_1}{b_0};
  	\edl{b_0};
  	\specl{below}{c_1}{b_1};
  	\only<1>{
	\link{b_1}{a_0};
  	\edl{a_0};}
  	\only<2->{
	\link{b_1}{c_0};
  	\edl{c_0};
  	\node[draw,dotted,fit=(c_0),label=above:$a_0\to c_0$]{};
  	}
    \end{tikzpicture}

\end{column}
\end{columns}

\only<1>{Reaching $a_1$ disables the reachability of $b_1$, and vice versa, $c_1$ is unreachable}
\only<2->{$c_1$ is reachable \textit{via} $a_1::b_1::c_1$}

\only<3->{
\begin{columns}
\begin{column}{0.5\textwidth}
LCG does not show \highlight{orders} of local states

\centering
$\Longrightarrow$ How to deal with this structure?
\end{column}
\begin{column}{0.5\textwidth}
\begin{tikzpicture}[aS,scale=0.9, every node/.style={scale=0.9}]  
  	\node[Aproc] (a_1) {$X$};
  	\node[Asol,right of=a_1] (a_1s) {};
  	\path (a_1) edge (a_1s);
    \node[Aproc, above right of=a_1s] (y){$Y$};
    \path (a_1s) edge (y);
    \node[Aproc,below right of=a_1s] (z){$Z$};
    \path (a_1s) edge (z);
\end{tikzpicture}
\end{column}
\end{columns}
}
\end{frame}

\begin{frame}{Two Possible Heuristics}
    Complete search \pause $\to$ State space explosion problem $\times$ \pause $\to$ Partial search
    
    \vspace{0.5cm}
    \pause     
\begin{columns}
\begin{column}{0.5\textwidth}
$\to$ Heuristics used to find a sequence of local states in the form $Z::Y::X$
\end{column}
\begin{column}{0.5\textwidth}
\centering
\begin{tikzpicture}[aS,scale=0.9, every node/.style={scale=0.9}]  
  	\node[Aproc] (a_1) {$X$};
  	\node[Asol,right of=a_1] (a_1s) {};
  	\path (a_1) edge (a_1s);
    \node[Aproc, above right of=a_1s] (y){$Y$};
    \path (a_1s) edge (y);
    \node[Aproc,below right of=a_1s] (z){$Z$};
    \path (a_1s) edge (z);
\end{tikzpicture}
\end{column}
\end{columns}

\vspace{0.5cm}
\pause     
\centering
    \begin{tabular}{c|c|c}
        &PermReach & ASPReach \\
        \hline
        Method &\makecell{Search all the \\permutations of branches}   &\makecell{Search all the possible\\ order of branches}\\
        \hline
        Runtime & $+$ & $-$ \\
        Conclusiveness& $-$&$+$
    \end{tabular}
     
\end{frame}

\begin{frame}{Example}
\begin{columns}
\begin{column}{0.45\textwidth}
    \begin{tikzpicture}[every node/.style={scale=0.9}]
\TSort{(0,0)}{a}{2}{l}
\TSort{(1.7,0)}{b}{2}{l}
\TSort{(3.7,0)}{c}{2}{l}
\path[local transitions]

	(c_0) edge node[auto] {\{$a_1, b_1$\}} (c_1)
    (a_0) edge[bend left] node[auto] {$\{b_0\}$} (a_1)
	(b_0) edge node[auto] {$\{\mathbf{c_0}\}$} (b_1)
;

\TState{a_0, b_0, c_0}


\end{tikzpicture}

\end{column}
\begin{column}{0.55\textwidth}
    \begin{tikzpicture}[aS]  
  	
  	  	\node[color=red,Aproc] (c_1) {$c_1$};\node[Asol,right of=c_1] (c_1s) {};\path (c_1) edge (c_1s);
  	\specl{above}{c_1}{a_1};
  	\link{a_1}{b_0};
  	\edl{b_0};
  	\specl{below}{c_1}{b_1};
  	\only<1>{
	\link{b_1}{a_0};
  	\edl{a_0};}
  	\only<2->{
	\link{b_1}{c_0};
  	\edl{c_0};
  	\node[draw,dotted,fit=(c_0),label=above:$a_0\to c_0$]{};
  	}
    \end{tikzpicture}

\end{column}
\end{columns}

\vspace{0.5cm}
Notation: $a\rhd b$ means $a$ appears in the sequence before $b$

\only<1>{
\begin{tabular}{lll}
Order constraints in LCG &$\Rightarrow$& $b_0\rhd a_1\rhd c_1$ , $a_0 \rhd b_1\rhd c_1$\\
Additional rule &$\Rightarrow$& $a_1 \rhd b_1$ and $b_1 \rhd a_1$
\end{tabular}

\vspace{0.5cm}
Contradiction in order, no solution
}

\only<2>{
\begin{tabular}{lll}
Order constraints in LCG &$\Rightarrow$& $b_0\rhd a_1\rhd c_1$ , $c_0 \rhd b_1\rhd c_1$\\
Additional rule &$\Rightarrow$& $a_1 \rhd b_1$
\end{tabular}

\vspace{0.5cm}
The only admissible order is $a_1\to b_1\to c_1$}
\end{frame}

\begin{frame}{Benchmarks: on Biological Examples}
Traditional model checkers: Mole NuSMV $\to$ \highlight{memory-out}

Pure static analyzer: Pint% \cite{folschette2015}

Small example: $\lambda$-phage, 4 components

Big examples: TCR (T-Cell Receptor, 95 components) and

EGFR (Epidermal Growth Factor Receptor, 106 components)

\small
    \centering
    \onslide<3->{\begin{tabular}{|c|c|c|c|}
    \hline
    Model    &  \multicolumn{3}{c|}{$\lambda$-phage}\\
    \hline
    Inputs    & 4 & Outputs& 4\\
    \hline
    Total tests&\multicolumn{3}{c|}{$2^4\times 4=64$}\\
    \hline
    Analyzer  &  Pint  &  \textbf{PermReach}   &\textbf{ASPReach}\\
    \hline
    Reachable & 36(56\%)& \multicolumn{2}{c|}{38(59\%)} \\
    \hline
    Unreachable&\multicolumn{3}{c|}{26(41\%)}\\
    \hline
    \textbf{Inconclusive} &\textcolor{red}{\textbf{2(3\%)}}&\multicolumn{2}{c|}{\textcolor{blue}{\textbf{0(0\%)}}}\\
    \hline
    Total time & \multicolumn{3}{c|}{$<1$s}\\
    \hline
\onslide<4->     Model    &  \multicolumn{3}{c|}{TCR}\\
    \hline
    Inputs    & 3 & Outputs& 5\\
    \hline
    Total tests&\multicolumn{3}{c|}{$2^3\times 5=40$}\\
    \hline
    Analyzer  &  Pint  &  \textbf{PermReach}   &\textbf{ASPReach}\\
    \hline
    Reachable & \multicolumn{3}{c|}{16(40\%)} \\
    \hline
    Unreachable&\multicolumn{3}{c|}{24(60\%)} \\
    \hline
    \textbf{Inconclusive} &\multicolumn{3}{c|}{\textcolor{blue}{\textbf{0(0\%)}}} \\
    \hline
    Total time &  7s     &0.85s  &  40s        \\
    \hline
\onslide<5->     Model    &  \multicolumn{3}{c|}{EGFR}\\
    \hline
    Inputs    & 13 & Outputs& 12\\
    \hline
    Total tests&\multicolumn{3}{c|}{$2^{13}\times 12=98,304$}\\
    \hline
    Analyzer  &  Pint  &  \textbf{PermReach}   &\textbf{ASPReach}\\
    \hline
    Reachable & 64,282(65.4\%)  & \multicolumn{2}{c|}{74,268(75.5\%)} \\
    \hline
    Unreachable&\multicolumn{3}{c|}{24,036(24.5\%)}\\
    \hline
    \textbf{Inconclusive} &\textcolor{red}{\textbf{9,986(10.1\%)}}&\multicolumn{2}{c|}{\textcolor{blue}{\textbf{0(0\%)}}}   \\
    \hline
    Total time & \textbf{9h50min}      & \textbf{15min31s}         & \textbf{3h46min} \\
    \hline
    \end{tabular}
    }
\end{frame}

\begin{frame}{Benchmarks: on Random Examples}
\begin{tikzpicture}[scale=0.725]
    \begin{axis}[xlabel=ABAN size,ylabel=runtime (s),legend pos=north west]
       \addplot[mark=triangle,color=red] coordinates{
        (100,0.029)
        (200,0.079)
        (300,0.209)
        (400,0.330)
        (500,0.729)
        (600,0.948)
        (700,1.379)
        (800,2.102)
        (900,3.416)
        (1000,4.58)
        };
        \addlegendentry{PermReach}
       \addplot[mark=square,color=blue] coordinates{
        (100,0.03)
        (200,0.08)
        (300,0.17)
        (400,0.40)
        (500,0.71)
        (600,1.04)
        (700,1.38)
        (800,2.63)
        (900,3.81)
        (1000,5.91)       
        };
        \addlegendentry{ASPReach}
    \end{axis}
\end{tikzpicture}
\begin{tikzpicture}[scale=0.725]
\begin{axis}[xlabel=ABAN density,ylabel=runtime (s),legend pos=north west]
\addplot[mark=triangle,color=red] coordinates {
        (1,0.01)
        (2,0.01)
        (3,0.02)
        (4,0.02)
        (5,0.32)
        (6,0.92)
        (7,1.24)
        (8,1.08)
        (9,1.02)
        (10,0.87)
        (11,0.75)
        (12,0.42)
    };
\addlegendentry{PermReach};
       \addplot[mark=square,color=blue] coordinates{
        (1,0.014)
        (2,0.015)
        (3,0.017)
        (4,0.017)
        (5,0.018)
        (6,0.2)
        (7,0.89)
        (8,1.987)
        (9,0.710)
        (10,0.71)
        (11,0.6)
        (12,0.36)    
        };
        \addlegendentry{ASPReach}
\end{axis}
\end{tikzpicture}
\end{frame}

\begin{frame}{Second Part: Learning and Revising Models}
    \begin{tikzpicture}[line,>=stealth]
    \onslide<+->{\node[ellipse, fill=blue!20] (tsd) {Time-series data};}
    \onslide<+->{\node[ellipse, fill=blue!20, right = of tsd] (model1) {Model};
    \draw[->] (tsd) -- (model1) node[midway, above] (lfit){Learning};
    }
    \onslide<+->{\node[ellipse, fill=blue!20, right = of model1] (reach) {Reachability properties};
    \draw[->] (model1) -- (reach) node[midway,label=inconsistent] (inconsis1){\huge$\times$};
    }
    \onslide<+->{\node[ellipse, fill=blue!20, above = of model1] (model2) {Model'};
    \draw[->] (model1) edge node[sloped, anchor=center, above]{Revise} (model2);
    }
    \onslide<+->{\draw[->] (model2) edge node[sloped, anchor=center]{} (reach);}
    \onslide<+->{\draw[->] (model2) edge node[sloped, anchor=center]{\huge$\times$} (tsd);}
    \onslide<+->{\node[ellipse, fill=blue!20, below = of model1] (model3) {Model''};
    \draw[->] (model1) edge[dashed,color=purple!50] node[sloped, anchor=center, above]{\highlight{Revise}} (model3);
    }
    \onslide<+->{\draw[->] (model3) edge node[sloped, anchor=center]{} (reach);}
    \onslide<+->{\draw[->] (model3) edge node[sloped, anchor=center]{} (tsd);}
\end{tikzpicture}

    \vspace{0.2cm}
    \begin{itemize}
    \item<+-> CRAC: Completion \textit{via} Reachability And Correlations
    \item<+-> M2RIT: Model Revision \textit{via} Reachability and Interpretation Transitions
\end{itemize}
\onslide<+->{
    \begin{tabular}{c|c|c}
         &CRAC & M2RIT \\
         \hline
         Learning phase & Correlation Coefficients & LFIT\\
         \hline
         Revising phase & \makecell{Reachability+\\candidate transitions} & \makecell{Reachability+\\time-series data}\\
         \hline
         Data type &Continuous & Discrete\\
         \hline
         Function &Inference & Precise learning\\
         \hline
         \makecell{Tolerance to\\ imprecision}&Yes&No
    \end{tabular}
}
\end{frame}

\begin{frame}{Learning Phase}
CRAC:

\vspace{0.2cm}
\begin{columns}
\begin{column}{0.5\textwidth}
\onslide<1->{
1.
\begin{tabular}{l|*{4}{l}}
$t$&0&1&2&3\\
\hline
$a$&2.01&2.51&1.97&1.17\\
$b$&0.74&0.87&0.78&0.33\\
\end{tabular} 
}

\onslide<3->{
3.
$r'=\kbordermatrix{\mbox{}&\Delta a&\Delta b\\
a&\text{N/A}&0.09\\
b&0.65&\text{N/A}\\
}$
}
\end{column}
\begin{column}{0.5\textwidth}
\onslide<2->{
2.
\begin{tabular}{c|c*{3}{S}}
$t$&1&2&3\\
\hline
$\Delta a$&0.5&-0.54&-0.8\\
$\Delta b$&0.13&-0.09&-0.45\\
\end{tabular}
}

\onslide<4->{
\begin{tikzpicture}[grn]
      \path[use as bounding box] (-0.3,-0.75) rectangle (4,.75);
      \node[inner sep=0] (b) at (2,0) {b};
      \node[inner sep=0] (a) at (0,0) {a};
      \node[draw=none] (lab) at (-0.6,0) {4.};    
      \path[->]
       (b) edge[bend right] node[elabel, above=-5pt] {} (a);
\end{tikzpicture}
}
\end{column}
\end{columns}

\vspace{0.5cm}
\onslide<5->{
M2RIT:

\vspace{0.2cm}
\begin{columns}
\begin{column}{0.4\textwidth}
Full transitions

$(a(t),b(t))\gets (a(t-1),b(t-1))$
\end{column}
\begin{column}{0.2\textwidth}
classed into

     $\Longrightarrow$ 
\end{column}
\begin{column}{0.4\textwidth}
    partial transitions
    
    $a_1\gets b_1$
\end{column}
\end{columns}
}
\end{frame}

\begin{frame}{Revising Phase}
    Methodology:
    
    \begin{center}
        \begin{tabular}{l|c|c}
        &Reachable &Unreachable\\
        \hline
        Knowledge& $R_K$ & $U_K$ \pause\\
        \hline
        Inferred model& $R_I$ & $U_I$\pause\\
        \hline
        Inconsistency (problem)& $R'_K=R_K\cap U_I$ & $U'_K=R_I\cap U_K$\pause\\
        Keep consistent with& $U_K$& $R_K$ \pause
    \end{tabular} 
    \end{center}
2 solutions:
\begin{itemize}
    \item General solution: cut set + completion set, complete but costly
    \item Heuristics, fast but not complete
\end{itemize}
\end{frame}

\begin{frame}{Cut Set}
    \begin{minipage}{0.6\textwidth}
\centering
\begin{tikzpicture}[apdotsimple/.style={apdot}]
\TSort{(0,0)}{a}{2}{l}
\TSort{(1.7,0)}{b}{2}{l}
\TSort{(3.4,0)}{c}{2}{l}
\TSort{(5.1,0)}{d}{2}{l}
% with delays
\path[local transitions]

	(a_0) edge node[auto] {\{$b_1, c_1$\}} (a_1)
	(c_0) edge node[auto] {\{$a_1$\}} (c_1)
	(a_0) edge[bend right] node[right] {\{$b_0$\}} (a_1)
	(b_1) edge[bend right] node[auto] {\{$d_1$\}} (b_0)
	(d_0) edge node[auto] {\{$b_1$\}} (d_1)
;
\TState{a_0, b_1, c_1,d_0}
\end{tikzpicture}
\end{minipage}\hfill
\pause
\begin{minipage}{0.4\textwidth}
\centering
\begin{tikzpicture}[aS]  
  	\node[Aproc] (a_1){\textcolor<8>{red}{$a_1$}};
  	\onslide<2-7,9->{\node[Asol, right = 0.5cm of a_1](a_1s){};	}
  	\onslide<8>{\node[Asol, right = 0.5cm of a_1,color=red](a_1s){};	}
    \node[Aproc, above right of = a_1s] (c_1){$c_1$};
    \node[Aproc, right of = a_1s] (b_1){$b_1$};
    \node[Asol, right of = c_1](c_1s){};
    \node[Assol, right of = c_1s](c_1st){\textcolor<7>{red}{$\varnothing$}};
    \node[Asol, right of = b_1](b_1s){};
    \node[Assol, right of = b_1s](b_1st){\textcolor<3>{red}{$\varnothing$}};
    \onslide<2-8,10->{\node[Asol,left of=c_1](c_1s1){};}
  	\onslide<9>{\node[Asol,left of=c_1,color=red](c_1s1){};}
  	\onslide<2-5,7->{\node[Asol,below right = 0.2cm and 0.3cm of a_1] (a_1s1){};}
  	\onslide<6>{\node[Asol,below right = 0.2cm and 0.3cm of a_1,color=red] (a_1s1){};}
    \node[Aproc, below right = 0.2cm and 0.3cm of a_1s1] (b_0){$b_0$};
    \onslide<2-4,6->{\node[Asol, right of = b_0](b_0s){};}
  	\onslide<5>{\node[Asol, right of = b_0,color=red](b_0s){};}
    \node[Aproc, right = 0.45cm of b_0s](d_1){$d_1$};
    \onslide<2,5->{\node[Asol, right of = d_1](d_1s){};}
  	\onslide<4>{\node[Asol, right of = d_1,color=red](d_1s){};}
    
    \path 
    (a_1s) edge (c_1)
    (a_1s) edge (b_1)
    (c_1) edge (c_1s)
    (c_1s) edge (c_1st)
    (b_1) edge (b_1s)
    (b_1s) edge (b_1st)
    (a_1) edge (a_1s)
    
    (c_1) edge (c_1s1)
    (c_1s1) edge (a_1)
    
    (a_1) edge (a_1s1)
    (a_1s1) edge (b_0)
    
    (b_0) edge (b_0s)
    (b_0s) edge (d_1)
    (d_1) edge (d_1s)
    (d_1s) edge (b_1)
    ; 
\end{tikzpicture}
\end{minipage}
\begin{columns}
\begin{column}{0.65\textwidth}
    \begin{tabular}{|c|c|l|}
\hline 
Node & Rank & $\mathbb{V}$ \\ 
\hline 
\textcolor<+>{red}{$\varnothing$ (of $b_1$) }& 1 & $\varnothing$ \\ 
\hline 
$b_1$ & 2 & $\{\{b_1\}\}$ \\ 
\hline 
\textcolor<+>{red}{$\{b_1\}\to d_1$} & 3 & $\{\{b_1\}\}$ \\ 
\hline 
$d_1$ & 4 & $\{\{b_1\},\{d_1\}\}$ \\ 
\hline 
\textcolor<+>{red}{$\{d_1\}\to b_0$ }& 5 & $\{\{b_1\},\{d_1\}\}$ \\ 
\hline 
$b_0$ & 6 & $\{\{b_0\},\{b_1\},\{d_1\}\}$ \\ 
\hline 
\textcolor<+>{red}{$\{b_0\}\to a_1$ }& 7 & $\{\{b_0\},\{b_1\},\{d_1\}\}$ \\ 
\hline 
\textcolor<+>{red}{$\varnothing$ (of $c_1$) }& 8 & $\varnothing$ \\ 
\hline 
$c_1$ & 9 & $\{\{c_1\}\}$ \\ 
\hline 
\textcolor<+>{red}{$\{b_1,c_1\}\to a_1$ }& 9 & $\{\{b_1\},\{c_1\}\}$ \\ 
\hline 
$a_1$ & 9 & $\{\{a_1\},\{b_1\},\{b_0,c_1\},\{c_1,d_1\}\}$ \\ 
\hline 
\textcolor<+>{red}{$\{a_1\}\to c_1$ }& 9 & $\{\{a_1\},\{b_1\},\{b_0,c_1\},\{c_1,d_1\}\}$ \\ 
\hline 
\end{tabular}    
\end{column}
\begin{column}{0.35\textwidth}
    By inhibiting the reachability of one of the sets in cut set $\{\{a_1\},\{b_1\},\{b_0,c_1\},\{c_1,d_1\}\}$, $a_1$ shall not be reachable.
    
\vspace{0.2cm}
Modification depends on the consistency with the result of learning phase.
\end{column}
\end{columns}    
\end{frame}
 
\begin{frame}{Completion Set}
    \begin{tikzpicture}[apdotsimple/.style={apdot},scale=0.9]
\TSort{(0,0)}{a}{2}{l}
\TSort{(1.5,0)}{b}{2}{l}
\TSort{(3,0)}{c}{2}{l}
\TSort{(4.5,0)}{d}{2}{l}
\TSort{(6,0)}{e}{2}{l}
% with delays
\path[local transitions]

	(a_0) edge node[auto] {\{$b_1, c_0$\}} (a_1)
	(c_0) edge node[auto] {\{$d_1$\}} (c_1)
	(a_0) edge[bend right] node[right] {\{$c_1, e_1$\}} (a_1)
;
\TState{a_0, b_0, c_0,d_0, e_0}
\end{tikzpicture}
\pause
\begin{tikzpicture}[aS,scale=0.9]  
  	\node[Aproc] (a_1){$a_1$};
  	\onslide<2-4,6->{\node[Asol, right = 0.5cm of a_1](a_1s){};}
  	\onslide<5>{\node[Asol, right = 0.5cm of a_1,color=red](a_1s){};}
    \node[Aproc, below right = 0.3cm and 0.5cm of a_1s] (b_1){$b_1$};
    \node[Aproc, right = 0.5cm of a_1s] (c_0){$c_0$};
    \node[Asol, right = 0.5cm of c_0](c_0s){};
    \node[Assol, right = 0.5cm of c_0s](c_0st){\textcolor<4>{red}{$\varnothing$}};
    \node[Assol, right = 0.5cm of b_1](b_1st){\textcolor<3>{red}{$\perp$}};
    \onslide<2-8,10->{\node[Asol, above right = 0.3cm and 0.5cm of a_1] (a_1s1){};}
  	\onslide<9>{\node[Asol, above right = 0.3cm and 0.5cm of a_1,color=red] (a_1s1){};}
    \node[Aproc, right = 0.5cm of a_1s1] (c_1){$c_1$};
    \node[Aproc, above right = 0.5cm of a_1s1] (e_1){$e_1$};
    \onslide<2-6,8->{\node[Asol, right = 0.5cm of c_1](c_1s){};}
  	\onslide<7>{\node[Asol, right = 0.5cm of c_1,color=red](c_1s){};}
    \node[Aproc, right = 0.5cm of c_1s] (d_1){$d_1$};
    \node[Assol, right = 0.5cm of d_1](d_1st){\textcolor<6>{red}{$\perp$}};
    \node[Assol, right = 0.5cm of e_1](e_1st){\textcolor<8>{red}{$\perp$}};
    \path 
    (a_1) edge (a_1s)
    (a_1s) edge (c_0)
    (a_1s) edge (b_1)
    (c_0) edge (c_0s)
    (c_0s) edge (c_0st)
    (b_1) edge (b_1st)
    %(b_1s) edge (b_1st)
    (a_1) edge (a_1s1)
    (a_1s1) edge (c_1)
    (c_1) edge (c_1s)
    (c_1s) edge (d_1)
    (d_1) edge (d_1st)
    (a_1s1) edge (e_1)
    (e_1) edge (e_1st)
    ; 
\end{tikzpicture}
    
\vspace{0.5cm}    
\begin{columns}
\begin{column}{0.65\textwidth}
    \begin{tabular}{|c|c|l|}
\hline 
Node & Rank & $\mathbb{V}$ \\ 
\hline 
$\perp$ (of $b_1$) & 1 & \textcolor<+>{red}{$\varnothing$ }\\ 
\hline 
$b_1$ & 2 & $\{\{b_1\}\}$ \\ 
\hline 
$\varnothing$ (of $c_0$) & 3 & \textcolor<+>{red}{$\{\varnothing\}$ }\\ 
\hline 
$c_1$ & 4 & $\{\varnothing\}$ \\ 
\hline 
$\{b_1,c_0\}\to a_1$ & 5 & \textcolor<+>{red}{$\{\{b_1\}\}$ }\\ 
\hline 
$\perp$ (of $d_1$) & 6 & \textcolor<+>{red}{$\varnothing$ }\\ 
\hline 
$d_1$ & 7 & $\{\{d_1\}\}$ \\ 
\hline 
$\{d_1\}\to c_1$ & 8 & $\{\{d_1\}\}$ \\ 
\hline 
$c_1$ & 9 & \textcolor<+>{red}{$\{\{c_1, d_1\}\}$ }\\ 
\hline 
$\perp$ (of $e_1$) & 10 & \textcolor<+>{red}{$\varnothing$ }\\ 
\hline 
$e_1$ & 11 & $\{\{e_1\}\}$ \\ 
\hline 
$\{c_1,e_1\}\to a_1$ & 12 & \textcolor<+>{red}{$\{\{c_1, e_1\}, \{d_1, e_1\}\}$ }\\ 
\hline 
$a_1$ & 13 & $\{\{a_1\}, \{b_1\}, \{c_1, e_1\}, \{d_1, e_1\}\}$ \\ 
\hline 
\end{tabular}   
\end{column}
\begin{column}{0.35\textwidth}
By assuring the reachability of one of the sets in completion set $\{\{a_1\},\{b_1\},\{b_0,c_1\},\{c_1,d_1\}\}$, $a_1$ shall not be reachable.

\vspace{0.2cm}
Modification depends on the consistency with the result of learning phase.
\end{column}
\end{columns} 
\end{frame} 
 
\begin{frame}{Heuristics}
\begin{columns}
\begin{column}{0.6\textwidth}
\begin{tikzpicture}[every node/.style={scale=0.9}]
\TSort{(0,0)}{a}{2}{l}
\TSort{(1.7,0)}{b}{2}{l}
\TSort{(3.4,0)}{c}{2}{l}
\TSort{(5.2,0)}{d}{2}{l}
\path[local transitions]
    (a_0) edge[bend right] node[right] {$\{b_1\}$} (a_1)
    (c_0) edge[bend right] node[right] {$\{b_0\}$} (c_1)
;

\onslide<1-4>{\path[local transitions]
(b_0) edge[bend right] node[right] {$\{c_0\}$} (b_1)
;
}
\onslide<5,6>{\path[local transitions]
(b_0) edge[bend right] node[right] {$\{c_0,\textcolor{blue}{a_1}\}$} (b_1)
;
}
\onslide<1-5>{\path[local transitions]
(a_0) edge node[auto] {$\{c_0,d_1\}$} (a_1)
;
}
\onslide<6>{\path[local transitions]
(a_0) edge node[auto] {$\{c_0,\textcolor{red}{\xcancel{d_1}}\}$} (a_1)
;
}
\onslide<7>{\path[local transitions]
(a_0) edge node[auto] {$\{c_0\}$} (a_1)
(b_0) edge[bend right] node[right] {$\{c_0,a_1\}$} (b_1)
;
}
\TState{a_0, b_0, c_0,d_0}


\end{tikzpicture}

\end{column}
\begin{column}{0.4\textwidth}
    $\alpha=\langle a_0,b_0,c_0,d_0\rangle$, $\omega=a_1$
    
	$U_K=\{(\alpha,b_1),(\alpha,d_1)\}$
	
	$R_K=\{(\alpha,a_1)\}$
\end{column}
\end{columns}

        \vspace{0.3cm}
        \centering
        \begin{tikzpicture}[aS]  
  	\node[Aproc] (a_1) {$a_1$};
  	\node[Asol,right of = a_1] (a_1s) {};
  	\node[Asol,below right of = a_1] (a_1s2) {};
  	\path (a_1) edge (a_1s);
  	\path[thick,dashed,color=blue]<4> (a_1) edge (a_1s2);
  	\path<1-3,5> (a_1) edge (a_1s2);
  	\node<1-4>[Aproc,right of = a_1s2] (d_1) {$d_1$};
  	\node[Aproc,below right of = a_1s2] (c_0) {$c_0$};
  	\node[Asol,right of = c_0] (c_0s) {};
  	\path<4>[thick,dashed,color=red] (a_1s2) edge node[color=black] {\LARGE $\times$} (d_1);
  	\path<1-3> (a_1s2) edge (d_1);
  	\path (a_1s2) edge (c_0);
  	\path (c_0) edge (c_0s);
  	\edl{c_0};
  	\node[Aproc,right of = a_1s] (b_1) {$b_1$};
  	\path (a_1s) edge (b_1);
  	\node[Asol, right of = b_1] (b_1s) {};
  	\path[thick,dashed,color=blue]<3-4> (b_1) edge (b_1s);
  	\path<1-2,5> (b_1) edge (b_1s);
  	\link{b_1}{c_0};
 	\edl{c_0};
 	\node<3->[Aproc,below right of = b_1s] (a_1) {$a_1$};
 	%\node[Asol,right of = c_1] (c_1s) {};
 	\path<3-4>[thick,dashed,color=green] (b_1s) edge (a_1);
 	\path<5-> (b_1s) edge (a_1);
 	%\path (c_1) edge (c_1s);
 	%\node[Aproc,right of = c_1s] (a_1) {$a_1$};
 	%\path (c_1s) edge (a_1);
	%\link{a_1}{c_1};
	%\link{c_1}{a_1};
 	%\edl{a_1};
\end{tikzpicture}
        
        \vspace{0.3cm}
     \begin{itemize}
        \item $L=\{\{(\alpha,a_1),(\alpha,b_1),(\alpha,d_1)\},\{(\alpha,b_1)\},\{(\alpha,d_1)\}\}$
        \item Start from $\{(\alpha,b_1)\}$ and $\{(\alpha,d_1)\}$
        \item $b_1\gets c_0$ can be specialized to $b_1\gets c_0\land a_1$ to make $b_1$ unreachable
        \item $a_1\gets d_1 \land c_0$ can only be generalized to $a_1 \gets c_0$ as $d_1\in U_K$
        \item Check the reachability of $(\alpha,a_1)$: reachable, finish
    \end{itemize}
\end{frame}

\begin{frame}{Conclusion}
\begin{itemize}
    \item Reachability Analysis
    \begin{itemize}
        \item PermReach and ASPReach: faster than traditional model checkers and more conclusive than pure static analysis $\to$ balance between complexity and completeness
        \item Deal with systems with up to $10^3$ automata
        \item $\xRightarrow{\text{?}}$ Hybrid analyzer is more performing in reachability analysis
    \end{itemize}
    \item Model Learning and Revision
    \begin{itemize}
        \item A comparatively new domain
        \item By only heuristics as metrics, only systems bisimilar to original systems
        \item General solution vs. heuristics $\to$ balance between complexity and completeness
        \item More information needed to obtain a precise model 
    \end{itemize}
\end{itemize}
\end{frame}

%\begin{frame}{Bibliography}
%    \bibliographystyle{plain}
%    \bibliography{bib}
%\end{frame}
\end{document}